% ! TEX program = xelatex
\documentclass[a5paper,fleqn,10pt]{article}
\usepackage[left=13mm,right=13mm,bottom=10mm,top=20mm]{geometry}
\usepackage[utf8]{inputenc}
\usepackage[T1]{fontenc}
\usepackage{amsmath}
\usepackage{amsthm}
\usepackage{thmtools}
\usepackage[UTF8]{ctex}
\usepackage{esint}
\usepackage{amssymb}
\usepackage{enumitem}
\usepackage{float}

\setlist[enumerate]{label={\roman*)}}
\pagestyle{empty}

% Add `\rm` to make contents within CTeX `amsthm` environments back Roman
\declaretheorem[title=例,postheadhook=\rm]{exmp}
\declaretheorem[title=定理,postheadhook=\rm]{thm}
\declaretheorem[title=解,postheadhook=\rm,style=remark,numbered=no]{solution}

\begin{document}

\begin{exmp}
	\begin{tabbing}
		\hspace{3em}a) 写出模9的一个完全剩余系, 它的每个数是奇数;\\
		\hspace{3em}b) 写出模9的一个完全剩余系, 它的每个数是偶数.
	\end{tabbing}
\end{exmp}
\begin{solution}
	\begin{tabbing}
		\hspace{2em}a) $\{-7,-5,-3,-1,1,3,5,7,9\}$;\\
		\hspace{2em}b) $\{-8,-6,-4,-2,0,2,4,6,8\}$.
	\end{tabbing}
\end{solution}

\begin{exmp}
	证明: 当$m>2$时, $0^2,1^2,\ldots,(m-1)^2$ 一定不是模 $m$ 的完全剩余系.
\end{exmp}
\begin{proof}
	由于 $(m-1)^2\equiv m^2-2m+1\equiv m(m-2)+1\equiv 1\pmod m$, 我们得出 $0,1,\ldots,m-1$ 中至多有
	$m-1$ 个互相不同余的数. 所以该数列不可能是模 $m$ 的完全剩余系.
\end{proof}

\begin{exmp}
	证明: 如果 $a^k\equiv b^k\pmod m$, $a^{k+1}\equiv b^{k+1}\pmod m$, 这里 $a,b,k,m$ 是整数, $k>0,m>0$,
	并且 $(a,m)=1$, 那么 $a\equiv b\pmod m$. 如果去掉 $(a,m)=1$ 这个条件, 结果仍成立吗?
\end{exmp}
\begin{proof}
	从 $a^k\equiv b^k\pmod m$, $a^{k+1}\equiv b^{k+1}\pmod m$ 开始, 将其两边同乘 $b$, 得到
	$a^kb\equiv b^{k+1}\equiv a^{k+1}\pmod m$, 经移项得 $a^k(a-b)\equiv 0\pmod m$, 即 $m\mid a^k(a-b)$.

	由于 $(a,m)=1$, 必然有 $(a^k,m)=1$ 成立. 所以 $m\mid (a-b)$, 即 $a\equiv b\pmod m$.
\end{proof}

\begin{exmp}
	$1^3+2^3+\cdots+(n-1)^3\equiv 0\pmod n$ 对于符合什么条件的正整数 $n$ 成立?
\end{exmp}
\begin{proof}
	已知平方和公式为 $\sum_{k=1}^nk^2=n(n+1)(2n+1)/6$, 首先求出立方和公式, 设 $S_n=\sum_{k=1}^nk^3$,
	$E_n=S_n-\int_0^nx^3\,{\rm d}x$, 我们有
	\begin{align*}
		\begin{aligned}
			E_n & =S_n-\int_0^nx^3\,{\rm d}x                              \\
			    & =\sum_{k=1}^n\left(k^3-\int_{k-1}^kx^3\,{\rm d}x\right) \\
			    & =\sum_{k=1}^n\left(\frac{3}{2}k^2-k+\frac{1}{4}\right)  \\
			    & =(2n^3+n^2)/4,
		\end{aligned}
		 &  &  & \text{那么} &
		\begin{aligned}
			S_n & =E_n+n^4/4               \\
			    & =\bigl(n(n+1)\bigr)^2/4.
		\end{aligned}
	\end{align*}
	所以 $1^3+2^3+\cdots+(n-1)^3=\bigl(n(n-1)\bigr)^2/4$. 题目便转为求使 $\bigl(n(n-1)\bigr)^2/4\equiv 0\pmod n$ 成立的 $n$,
	即 $n\mid\bigl(n(n-1)\bigr)^2/4$. 由此得存在整数 $k$ 使得 $kn=\bigl(n(n-1)\bigr)^2/4$. 经移项得 $4k=n(n-1)^2$, 即
	$4\mid n(n-1)^2$.
	\begin{enumerate}
		\item 当 $n=4k$ 时, $n(n-1)^2=4k(4k-1)$, 能被4整除;
		\item 当 $n=2k+1$ 时, $n(n-1)^2=4k^2(2k+1)$, 能被4整除;
		\item 当 $n=4k+2$ 时, $n(n-1)^2=(4k+2)(4k+1)^2=4(16k^3+16k^2+5k)+2$. 故 $n(n-1)^2\equiv 2\pmod 4$, 它不能被4整除.
	\end{enumerate}
	所以, 当 $n$ 不取形式为 $4k+2$ 形式的整数时, 原式成立.
\end{proof}

\begin{exmp}
	证明: 如果 $p$ 是素奇数, 那么 $1^2\cdot 3^2\cdots(p-4)^2\cdot (p-2)^2\equiv (-1)^{(p+1)/2}\pmod p$.
\end{exmp}
\begin{proof}
	将平方项拆分成正负两项相乘,
	\begin{align*}
		 & 1^2\cdot 3^2\cdots (p-4)^2\cdot(p-2)^2                                                        \\
		 & \equiv 1\cdot(-1)\cdot 3\cdot(-3)\cdots(p-4)\cdot(4-p)\cdot(p-2)\cdot(2-p)\cdot(-1)^{(p-1)/2} \\
		 & \equiv 1\cdot(p-1)\cdot 3\cdot(p-3)\cdots(p-4)\cdot 4\cdot(p-2)\cdot 2\cdot(-1)^{(p-1)/2}     \\
		 & \equiv 1\cdot 2\cdot 3\cdots(p-2)\cdot(p-1)\cdot(-1)^{(p-1)/2}                                \\
		 & \equiv(-1)^{(p+1)/2}\pmod p.\qedhere
	\end{align*}
\end{proof}

\begin{thm}\label{thm:sqrt}
	若 $x,y$ 是整数, $p$ 是素数, 且 $x^2\equiv y^2\pmod p$, 那么 $x\equiv\pm y\pmod p$.
\end{thm}
\begin{proof}
	由 $x^2\equiv y^2\pmod p$ 知 $p\mid(x^2-y^2)=(x-y)(x+y)$. 因为 $p$ 是素数, 我们有 $p\mid(x-y)$ 或 $p\mid(x+y)$,
	故 $x\equiv y\pmod p$ 或 $x\equiv -y\pmod p$.
\end{proof}

\begin{exmp}
	运用 Wilson 理论证明: 如果 $p$ 是素数, 并且 $p\equiv 1\pmod 4$, 那么同余式 $x^2\equiv -1\pmod p$ 就有两不同余解
	$x\equiv\pm\left(\frac{p-1}{2}\right)!\pmod p$.
\end{exmp}
\begin{proof}
	由 $x^2\equiv -1\pmod p$ 得
	\begin{align*}
		x^2 & \equiv 1\cdot 2\cdots(p-2)\cdot(p-1)                                              \\
		    & \equiv 1\cdot 2\cdots\frac{p-1}{2}\cdot\frac{p+1}{2}\cdots(p-2)\cdot(p-1)         \\
		    & \equiv \left(\frac{p-1}{2}\right)^2\cdots 2^2\cdot 1^2\cdot(-1)^{(p-1)/2}\pmod p.
	\end{align*}
	由于 $p\equiv 1\pmod 4$, 存在整数 $k$, 使得 $n=4k+1$ 成立, 则 $(p-1)/2=2k$ 是偶数, $(-1)^{(p-1)/2}=1$.
	再由定理\ref{thm:sqrt} 可得 $x\equiv\pm\left(\frac{p-1}{2}\right)!\pmod p$.
\end{proof}

\begin{exmp}
	证明: 如果$c_1,c_2,\ldots,c_{\phi(m)}$ 是模 $m$ 的简化剩余系, 其中 $m\geq 3$, 那么 $c_1+c_2+\cdots+c_{\phi(m)}\equiv 0\pmod m$.
\end{exmp}
\begin{proof}
	如果 $c$ 在 $m$ 的简化剩余系中, 那么 $n-c$ 也在 $m$ 的简化剩余系中, 因为 $(c,n)=(n-c,n)$.
	但是不可能出现 $c=n-c$ 的情况, 否则 $2c=n$, 则 $(c,n)=c$, 不符合 $c$ 与 $n$ 互质的要求. 所以,
	可以将简化剩余系中 $c_i$ 和 $n-c_i$ 两两配对, 使各对中两个元素的和为 $n$, 进而使所有元素的总和
	与 $0$ 模 $m$ 同余.
\end{proof}

\begin{exmp}
	设 $p$ 是奇素数, $k$ 是正整数. 证明: 同余式 $x^2\equiv 1\pmod{p^k}$ 正好有两个不同余的解 $x\equiv\pm 1\pmod{p^k}$.
\end{exmp}
\begin{proof}
	假设 $x^2\equiv 1\pmod{p^k}$, 那么 $x^2-1\equiv(x-1)(x+1)\equiv 0\pmod{p^k}$, 这要求了 $p^k\mid(x-1)(x+1)$.
	由于 $(x+1)-(x-1)=2$, 并且 $p$ 是奇素数($p^k\geq 3$), 所以 $p^k$ 只能整除 $x-1$ 和 $x+1$ 中的一个.
	故原同余式恰好有两个解, 即 $x\equiv\pm 1\pmod{p^k}$.
\end{proof}

\begin{exmp}
	证明: $k>2$ 时, 同余式 $x^2\equiv 1\pmod{2^k}$ 恰好有四个不同余的解, 它们是 $x\equiv\pm 1\;\text{or}\pm\!(1+2^{k-1})\pmod{2^k}$;
	$k=1$ 时, 该同余式有一个解; $k=2$ 时, 该同余式有两个不同余的解.
\end{exmp}
\begin{proof}
	假设 $x^2\equiv 1\pmod{2^k}$, 那么 $x^2-1\equiv(x-1)(x+1)\equiv 0\pmod{2^k}$, 这要求了 $2^k\mid(x-1)(x+1)$.
	注意到 $x-1$ 和 $x+1$ 都是偶数, 并且 $(x+1)-(x-1)=2$, 所以这两数中的一个不能被 $2$ 以上的 $2$ 的正幂次整除. 故
	$2^{k-1}\mid x-1$ 且 $2\mid x+1$ 或 $2^{k-1}\mid x+1$ 且 $2\mid x-1$. 由此得出原同余式的解有形式 $x=t2^{k-1}+1$
	和 $x=t2^{k-1}-1$.
	\begin{enumerate}
		\item $k>2$ 时, 取 $t=0\;\text{or}\;1$, 得到四个不同余的解 $x\equiv\pm 1\;\text{or}\pm\!(1+2^{k-1})\pmod{2^k}$;
		\item $k=1$ 时, 取 $t=0$, 只得到一个解 $x\equiv 1\pmod 2$;
		\item $k=2$ 时, 取 $t=0$, 得到两个不同余的解 $x\equiv\pm 1\pmod{4}$.\qedhere
	\end{enumerate}
\end{proof}

\begin{exmp}\label{exmp:unique_solution1}
	证明: 同余方程组 $x\equiv a_1\pmod{m_1}$, $x\equiv a_2\pmod{m_2}$ 有解当且仅当 $(m_1,m_2)\mid a_1-a_2$. 并证明若有解, 则
	该解模 $[m_1,m_2]$ 是唯一的.
\end{exmp}
\begin{proof}
	若第一条同余式成立, 当且仅当存在整数 $k$, 使得 $x=a_1+m_1k$ 成立. 将其代入第二条同余式, 得到 $a_1+m_1k\equiv a_2\pmod{m_2}$, 或者
	\begin{equation}\label{eqn:cong1}
		m_1k\equiv a_2-a_1\pmod{m_2}.
	\end{equation}
	如果该同余式有解, 那么原同余方程组有解, 这当且仅当 $(m_1,m_2)\mid a_1-a_2$.

	现在假定 $(m_1,m_2)\mid a_1-a_2$, 则 (\ref{eqn:cong1}) 式恰好有 $(m_1,m_2)$ 个解. 将 (\ref{eqn:cong1}) 式的一个解记为 $k_0$, 这些解为
	$k_0+jm_2/(m_1,m_2)$, $j=0,1,\ldots,(m_1,m_2)-1$. 那么根据 $x=a_1+m_1k$, 得出原同余方程组的解为
	$a_1+m_1k_0+jm_1m_2/(m_1,m_2)=a_1+m_1k_0+j[m_1,m_2]$, $j=0,1,\ldots,(m_1,m_2)-1$, 它们模 $[m_1,m_2]$ 同余. 所以原方程组的解
	模 $[m_1,m_2]$ 是唯一的.
\end{proof}

\begin{exmp}
	求解同余式 $59x\equiv 20\pmod{91}$. 使用欧几里得算法和中国剩余定理.
\end{exmp}
\begin{solution}
	a) 使用欧几里得算法.

	求解原同余式, 首先求解 $59x-91y=(59,91)$. 使用欧几里得算法(递推式 $s_j=s_{j-2}-q_{j-1}s_{j-1}$
	和 $t_j=t_{j-2}-q_{j-1}t_{j-1}$, 其中 $q_j$ 代表第 $j$ 次使用带余除法的商):

	\begin{minipage}{0.6\linewidth}%
		\begin{tabular}{c|rrrrrr}
			$j$ & $r_j$ & $r_{j+1}$ & $q_{j+1}$ & $r_{j+2}$ & $s_j$ & $t_j$ \\
			\hline
			0   & 91    & 59        & 1         & 32        & 1     & 0     \\
			1   & 59    & 32        & 1         & 27        & 0     & 1     \\
			2   & 32    & 27        & 1         & 5         & 1     & $-1$  \\
			3   & 27    & 5         & 5         & 2         & $-1$  & 2     \\
			4   & 5     & 2         & 2         & 1         & 2     & $-3$  \\
			5   & 2     & 1         & 2         & 0         & $-11$ & 17    \\
			6   &       &           &           &           & 24    & $-37$
		\end{tabular}%
	\end{minipage}
	\parbox{0.3\linewidth}{这说明了该方程的特解为 $x_0=-37,y_0=-24$.}
	\\\\
	所以 $59\cdot(-37)\equiv 59\cdot 54\equiv 1\pmod{91}$, 即 $59$ 模 $91$ 的乘法逆元为 $54$.
	代入原方程, $x\equiv 20\cdot(59)^{-1}\equiv 20\cdot 54\equiv 79\pmod{91}$.

	b) 使用中国剩余定理.

	首先求出 $x$ 模 $91$ 的乘法逆元. 由于 $91=7\cdot 13$, 得出同余方程组
	\begin{align*}
		\begin{aligned}
			 & 59x\equiv 3x\equiv 6\pmod 7   \\
			 & 59x\equiv 7x\equiv 7\pmod{13}
		\end{aligned} &  &  & \text{或者} &
		\begin{aligned}
			 & x\equiv 2\pmod 7   \\
			 & x\equiv 1\pmod{13}
		\end{aligned}
	\end{align*}
	根据例 \ref{exmp:unique_solution1}, 该方程组模 $91$ 有唯一解 且该解就是原同余式的解. 根据中国剩余定理, 我们有 $M=7\cdot 13=91$, $M_1=91/7=13$,
	$M_2=91/13=7$. 要确定 $y_1$ 和 $y_2$, 解 $13y_1\equiv 1\pmod 7$ 得出 $y_1=6$, 解 $7y_2\equiv 1\pmod{13}$ 得出 $y_2=2$.
	所以该同余方程组的解为 $x=a_1M_1y_1+a_2M_2y_2=2\cdot 13\cdot 6+1\cdot 7\cdot 2\equiv 79\pmod{91}$.
\end{solution}

\end{document}
