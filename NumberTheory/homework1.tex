% ! TEX program = xelatex
\documentclass[a4paper,fleqn]{article}
\usepackage[left=1in,right=1in,bottom=20mm,top=25mm]{geometry}
\usepackage[utf8]{inputenc}
\usepackage[T1]{fontenc}
\usepackage{amsmath}
\usepackage{amsthm}
\usepackage{thmtools}
\usepackage[UTF8]{ctex}
\usepackage{esint}
\usepackage{amssymb}
\usepackage{enumitem}

\setlist[enumerate]{label={\roman*)}}

% Add `\rm` to make contents within CTeX `amsthm` environments back Roman
\declaretheorem[title=例,postheadhook=\rm]{exmp}
\declaretheorem[title=解,postheadhook=\rm,style=remark,numbered=no]{solution}

\begin{document}

\begin{exmp}
    求证: 每个合数一定有素因子.
    \begin{proof}
        假设 $n$ 是一个正合数, $p$ 是 $n$ 的一个大于 $1$ 的最小正因数. 如果 $p$ 不是素数, 则存在整数 $1<q<p$,
        使得 $q\mid p$. 根据整除的传递性, 由于 $p\mid n$, 有 $q\mid n$. 这与 $p$ 是 $n$ 的最小正因数矛盾.
        所以 $p$ 是素数, 合数 $n$ 必有素因子.
    \end{proof}
\end{exmp}

\begin{exmp}
    求证: 若 $5\mid n$, $11\mid n$, 则 $55\mid n$.
    \begin{proof}
        由条件得 $n=5q_1$, $n=11q_2$, 其中 $q_1$, $q_2$ 为整数. 联立两式得 $(11-5\cdot2)n=55(q_1-q_2\cdot2)$,
        其中 $q_1-q_2\cdot2$ 也是整数, 所以 $55\mid n$.
    \end{proof}
\end{exmp}

\begin{exmp}
    求证: 每个奇整数的平方必有 $8k+1$ 的形式.
    \begin{proof}
        每个奇整数的平方必有形式 $(2n+1)^2$, 其中 $n$ 是整数. 若其也有 $8k+1$ 的形式, 则
        \begin{align*}
            (2n+1)^2  & =8k+1 \\
            4n^2+4n+1 & =8k+1 \\
            n(n+1)    & =2k.
        \end{align*}
        为使 $k$ 为整数, 需要 $2\mid n(n+1)$. 而 $n$ 和 $n+1$ 中一定有一数是 $2$ 的倍数, 故结论成立.
    \end{proof}
\end{exmp}

\begin{exmp}
    求 $1414$ 和 $666$ 的最大公因数, 并求出它们的线性表达式.
    \begin{solution}
        使用拓展欧几里得算法:
        \begin{align*}
            1414 & =666\cdot2+82 & 82 & =1414+666\cdot(-2)          \\
            666  & =82\cdot8+10  & 10 & =1414\cdot(-8)+666\cdot17   \\
            82   & =10\cdot8+2   & 2  & =1414\cdot65+666\cdot(-138) \\
            10   & =2\cdot5+0.
        \end{align*}
        得到它们的最大公因数是 $2$, 并且 线性表达式为 $1414\cdot(65+666k)+666\cdot(-138-1414k)=2$, 其中 $k$ 取任何整数.
    \end{solution}
\end{exmp}

\begin{exmp}
    求证: $\sqrt{2}$, $\sqrt{7}$, $\sqrt{17}$ 都不是有理数.
    \begin{proof}
        假设 $p$ 是素数, 且 $\sqrt{p}$ 是有理数, 则 $\sqrt{p}=m/n$, 其中 $m$, $n$ 是互质的正整数.
        \begin{enumerate}
            \item 经变换得 $pn^2=m^2$, 根据整除的定义, 有 $p\mid m^2$. 又因为 $p$ 是素数, 有 $p\mid m$.
            \item 由 $p\mid m$ 得, 存在整数 $k$, 使得 $m=pk$.
            \item 将 $m=pk$ 代入 $pn^2=m^2$, 得 $n^2=pk^2$, 则类似步骤 $(1)$, 有 $p\mid n$,
            \item 由 $(1)(3)$ 得 $m$ 与 $n$ 都有因数 $p>1$, 这与 $m$, $n$ 互质矛盾.
            \item 假设有误, 故 $\sqrt{p}$ 是无理数. 而 $2$, $7$, $17$ 都是素数, 所以 $\sqrt{2}$, $\sqrt{7}$, $\sqrt{17}$ 都不是有理数.\qedhere
        \end{enumerate}
    \end{proof}
\end{exmp}

\begin{exmp}
    求证: 形如 $4k+3$ 的素数有无穷多个.
    \begin{proof}
        首先, 要证明形如 $4k+3$ 的正整数必含有形如 $4k+3$ 的素因数: 任意奇素数都只能写成 $4k+1$ 或 $4k+3$ 两种形式. 若将两个形如 $4k+1$ 的数 $4n_1+1$, $4n_2+1$ 相乘, 即
        \begin{align*}
            (4n_1+1)(4n_2+1) & =16n_1n_2+4n_1+4n_2+1  \\
                             & =4(4n_1n_2+n_1+n_2)+1.
        \end{align*}
        经过数学归纳得, 有限个形如 $4k+1$ 的数的乘积仍为形式为 $4k+1$ 的数. 因此, 将形如 $4k+3$ 的整数分解为若干个素因数的乘积时, 这些素因数中必须含有形如 $4k+3$ 的素数.

        假设所有形如 $4k+3$ 的素数 $p_1,p_2,\ldots,p_n$ 都不大于一正整数 $N$. 令 $q=4(p_1p_2\cdots p_n)-1$. 那么任何 $p_i\;(i=1,2,\ldots,n)$ 都不是
        $q$ 的素因数, 否则将得到 $p_i\mid1$, 这不可能.

        若 $q$ 是素数, 由 $q=4(p_1p_2\cdots p_n)-1=4(p_1p_2\cdots p_n-1)+3$ 得知它是 $4k+3$ 形式的素数, 并且 $q>N$; 若 $q$ 不是素数, 由上述推理得其必含有形如 $4k+3$
        的素因数, 而且任何 $p_i\;(i=1,2,\ldots,n)$ 都不是 $q$ 的素因数. 所以 $q$ 是形如 $4k+3$ 且一定大于 $N$ 的素数.
    \end{proof}
\end{exmp}

\begin{exmp}
    设 $p$ 是合数 $n$ 的最小素因数. 求证: 若 $p>n^{1/3}$, 则 $n/p$ 是素数.
    \begin{proof}
        根据算术基本定理, $n$ 能被唯一分解为一系列素数的乘积, 即
        \[
            n=p_1p_2\cdots p_k\quad(1<p_1\leq p_2\leq\cdots\leq p_k<n,\;k\geq1).
        \]
        经过放缩, 得到 $p_1^k\leq n$, 即 $p_1\leq n^{1/k}$. 若 $k\geq3$, 则 $p_1\leq n^{1/3}$. 但根据题意, 有 $p_1>n^{1/3}$,
        从而有 $k<3$, 即 $k$ 只能取值 $1$ 或 $2$. 接下来分别考虑 $k$ 的取值情况. 若 $k$ 取 $1$, 那么 $n=p_1$ 是素数, 不符合题意.
        所以 $k$ 只能取 $2$. 那么 $n=p_1p_2$, 其中 $p_1$ 是 $n$ 的最小素因数, 故 $p=p_1$. 所以 $n/p=n_2$, 根据上述分解, 它是素数.
    \end{proof}
\end{exmp}

\end{document}