% ! TEX program = xelatex
\documentclass[a5paper,fleqn,10pt]{article}
\usepackage[left=13mm,right=13mm,bottom=10mm,top=20mm]{geometry}
\usepackage[utf8]{inputenc}
\usepackage[T1]{fontenc}
\usepackage{amsmath}
\usepackage{amsthm}
\usepackage{thmtools}
\usepackage[UTF8]{ctex}
\usepackage{esint}
\usepackage{amssymb}
\usepackage{enumitem}

\setlist[enumerate]{label={\roman*)}}
\pagestyle{empty}

% Add `\rm` to make contents within CTeX `amsthm` environments back Roman
\declaretheorem[title=例,postheadhook=\rm]{exmp}
\declaretheorem[title=解,postheadhook=\rm,style=remark,numbered=no]{solution}

\begin{document}

\begin{exmp}
    求证: 每个合数一定有素因子.
    \begin{proof}
        假设 $n$ 是一个正合数, $p$ 是 $n$ 的一个大于 $1$ 的最小正因数. 如果 $p$ 不是素数, 则存在整数 $1<q<p$,
        使得 $q\mid p$. 根据整除的传递性, 由于 $p\mid n$, 有 $q\mid n$. 这与 $p$ 是 $n$ 的最小正因数矛盾.
        所以 $p$ 是素数, 且合数 $n$ 必有素因子.
    \end{proof}
\end{exmp}

\begin{exmp}
    求证: 每个奇整数的平方必有 $8k+1$ 的形式.
    \begin{proof}
        每个奇整数的平方有形式 $(2n+1)^2$, 其中 $n$ 是整数. 若其也有 $8k+1$ 的形式, 则
        \begin{align*}
            (2n+1)^2  & =8k+1 \\
            4n^2+4n+1 & =8k+1 \\
            n(n+1)    & =2k.
        \end{align*}
        为使 $k$ 为整数, 需要 $2\mid n(n+1)$. 而 $n$ 和 $n+1$ 中一定有一数是 $2$ 的倍数, 所以结论成立.
    \end{proof}
\end{exmp}

\begin{exmp}
    求证: 若 $5\mid n,\;11\mid n$, 则 $55\mid n$.
    \begin{proof}
        由题意得, 必存在正整数 $q_1,q_2$ 使得 $n=5q_1,n=11q_2$. 将两边同时乘以 $11$ 或 $5$, 得到
        $11n=55q_1,5n=55q_2$. 由 $n=11n-(5n)\cdot 2=55(q_1-q_2\cdot 2)$ 可以看出 $55\mid n$.
    \end{proof}
\end{exmp}

\begin{exmp}
    设 $p$ 是合数 $n$ 的最小素因数. 求证: 若 $p>n^{1/3}$, 则 $n/p$ 是素数.
    \begin{proof}
        根据算术基本定理, $n$ 能被唯一分解为一系列素数的乘积, 即
        \[
            n=p_1p_2\cdots p_k\quad(1<p_1\leq p_2\leq\cdots\leq p_k<n,\;k\geq1).
        \]
        经过放缩, 得到 $p_1^k\leq n$, 即 $p_1\leq n^{1/k}$. 若 $k\geq3$, 则 $p_1\leq n^{1/3}$. 但根据题意, 有 $p_1>n^{1/3}$,
        从而有 $k<3$, 即 $k$ 只能取值 $1$ 或 $2$. 接下来分别考虑 $k$ 的取值情况. 若 $k$ 取 $1$, 那么 $n=p_1$ 是素数, 不符合题意.
        所以 $k$ 只能取 $2$. 那么 $n=p_1p_2$, 其中 $p_1$ 是 $n$ 的最小素因数, 故 $p=p_1$. 所以 $n/p=p_2$, 是一个素数.
    \end{proof}
\end{exmp}

\begin{exmp}
    求证: 形如 $4k+3$ 的素数有无穷多个.
    \begin{proof}
        首先, 要证明形如 $4k+3$ 的正整数必含有形如 $4k+3$ 的素因数: 任意奇素数都只能写成 $4k+1$ 或 $4k+3$ 两种形式. 若将两个形如 $4k+1$ 的数 $4n_1+1$ 和 $4n_2+1$ 相乘, 即
        \begin{align*}
            (4n_1+1)(4n_2+1) & =16n_1n_2+4n_1+4n_2+1  \\
                             & =4(4n_1n_2+n_1+n_2)+1.
        \end{align*}
        经过归纳, 可知有限个形如 $4k+1$ 的数的乘积仍为形式为 $4k+1$ 的数. 因此, 将形如 $4k+3$ 的整数分解为若干个素因数的乘积时, 这些素因数中必须含有形如 $4k+3$ 的素数.

        假设所有形如 $4k+3$ 的素数 $p_1,p_2,\ldots,p_n$ 都不大于一正整数 $N$. 令 $q=4(p_1p_2\cdots p_n)-1$. 那么任何 $p_i\;(i=1,2,\ldots,n)$ 都不是
        $q$ 的素因数, 否则将得到 $p_i\mid1$, 这不可能.

        若 $q$ 是素数, 由 $q=4(p_1p_2\cdots p_n)-1=4(p_1p_2\cdots p_n-1)+3$ 得知它是 $4k+3$ 形式的素数, 并且 $q>N$; 若 $q$ 不是素数, 由上述推理得其必含有形如 $4k+3$
        的素因数, 而且任何 $p_i\;(i=1,2,\ldots,n)$ 都不是 $q$ 的素因数. 所以 $q$ 是形如 $4k+3$ 且一定大于 $N$ 的素数.
    \end{proof}
\end{exmp}

\begin{exmp}
    设 $m>n$ 是正整数. 证明 $2^n-1\mid 2^m-1$ 的充要条件是 $n\mid m$. 以任一正整数 $a>2$ 代替 $2$, 结论仍成立吗?
    \begin{proof}
        首先证明充分性. 由 $n\mid m$ 可知, 存在正整数 $k$ 使得 $m=kn$. 所以
        \begin{align*}
            2^m-1&=2^{kn}-1\\
            &=(2^n-1)(2^{(k-1)n}+2^{(k-2)n}+\cdots+2^n+1).
        \end{align*}
        可以看出 $2^n-1\mid 2^m-1$.

        接下来证明必要性. 使用带余除法, 得到 $m=nq+r\;(0\leq r<n)$. 所以
        \begin{align*}
            2^m-1&=2^{kn+r}-1=2^{kn}2^r-1\\
            &=2^{kn}2^r-2^r+2^r-1\\
            &=2^r(2^{kn}-1)+2^r-1\\
            &=N(2^n-1)+2^r-1,
        \end{align*}
        其中 $N$ 是某个整数, 因为之前证明了 $2^n-1\mid 2^{km}-1$. 根据 $2^n-1\mid 2^m-1$,
        可知 $2^n-1\mid 2^r-1$. 但由于 $2^r-1<2^n-1$, 必须有 $2^r-1=0$, 即 $r=0$. 代入 $m=nq+r$
        中, 可以得到 $n\mid m$.
    \end{proof}
\end{exmp}

\begin{exmp}
    设奇数 $a>2$. 设使得 $a\mid 2^d-1$ 的最小正整数 $d=d_0$. 证明: $2^d$ 被 $a$ 整除后, 所可能取到的
    不同的最小非负余数有 $d_0$ 个.
    \begin{proof}
        由于 $a\mid 2^{d_0}-1$, 那么对于所有 $1\leq d<d_0$, 都有 $a\nmid 2^d-1$, 否则不满足 $d_0$
        是满足要求的最小正整数.
        
        接下来要说明, 由 $2^d-1\;(d=1,2,\ldots,d_0)$ 组成的序列中, 每个数除以
        $a$ 得到的余数两两不相同. 假设存在正整数 $1\leq i<j\leq d_0$ 使得 $2^i-1$ 和 $2^j-1$
        除以 $a$ 得到的余数都为 $r\;(0\leq r<a)$. 也就是 $2^i-1=aq_1+r,\;2^j-1=aq_2+r$.
        将两式相减, 得到 $2^j-2^i=a(q_2-q_1)$, 即 $a\mid 2^j-2^i=2^{j-i}(2^i-1)$. 由于
        $a\nmid 2^i-1$, 这说明 $a\mid 2^{j-i}$, 进而 $a$ 是偶数. 我们知道 $a\mid 2^{d_0}-1$,
        但是偶数不能整除奇数, 这造成了矛盾.

        另外, 我们可以发现
        \begin{align*}
            2^{d+d_0}-1&=2^d2^{d_0}-1\\
            &=2^d2^{d_0}-2^d+2^d-1\\
            &=2^d(2^{d_0}-1)+2^d-1\\
            &=aN+2^d-1,
        \end{align*}
        其中 $N$ 是某个整数. 这可以说明对于所有正整数 $d$, 都有 $a$ 整除 $2^d-1$ 的余数和 $a$ 整除
        $2^{d+d_0}-1$ 的余数相同. 所以数列 $2^d-1\;(d=1,2,\ldots,d_0)$ 中每个数除以 $a$ 得到
        的余数就是所有可能取到的 $d_0$ 个余数.
    \end{proof}
\end{exmp}

\begin{exmp}
    求 $1414$ 和 $666$ 的最大公因数, 并求出它们的线性表达式.
    \begin{solution}
        使用拓展欧几里得算法:
        \begin{align*}
            1414 & =666\cdot2+82 & 82 & =1414+666\cdot(-2)          \\
            666  & =82\cdot8+10  & 10 & =1414\cdot(-8)+666\cdot17   \\
            82   & =10\cdot8+2   & 2  & =1414\cdot65+666\cdot(-138) \\
            10   & =2\cdot5+0.
        \end{align*}
        得到它们的最大公因数是 $2$. 线性表达式为 $1414\cdot(65+666k)+666\cdot(-138-1414k)=2$, 其中 $k$ 取任何整数.
    \end{solution}
\end{exmp}

\begin{exmp}
    求证: $\sqrt{2},\sqrt{7},\sqrt{17}$ 都不是有理数.
    \begin{proof}
        假设 $p$ 是素数, 且 $\sqrt{p}$ 是有理数, 则 $\sqrt{p}=m/n$, 其中 $m,n$ 是互质的正整数.
        经变换得 $pn^2=m^2$. 根据整除的定义, 有 $p\mid m^2$. 因为 $p$ 是素数, 又有 $p\mid m$, 即存在
        整数 $k$, 使得 $m=pk$. 将 $m=pk$ 代入 $pn^2=m^2$ 得到 $n^2=pk^2$. 重复上述步骤, 可以
        发现 $p\mid n$.
        
        这说明 $m$ 与 $n$ 都有质因数 $p$, 与 $m,n$ 互质矛盾. 所以假设有误, 故 $\sqrt{p}$ 是无理数.
        而 $2,7,17$ 都是素数, 所以 $\sqrt{2},\sqrt{7},\sqrt{17}$ 都不是有理数.\qedhere
    \end{proof}
\end{exmp}

\begin{exmp}
    设整数 $a>b>0,\;n>1$. 证明: $a^n-b^n\nmid a^n+b^n$.
    \begin{proof}
        假设 $a^n-b^n\mid a^n+b^n$. 那么
        \[
            \frac{a^n+b^n}{a^n-b^n}=1+\frac{2b^n}{a^n-b^n}=1+\frac{2}{(a/b)^n-1}.
        \]
        这说明了 $(a/b)^n$ 只能取 $2$ 或 $3$. 但这在 $n>1$ 时不可能成立, 与假设矛盾.
    \end{proof}
\end{exmp}

\end{document}